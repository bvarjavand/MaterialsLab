\documentclass{article}
\usepackage[utf8]{inputenc}
\usepackage{graphicx}

\author{Bijan Varjavand}
\title{Lab Notebook}
\date{September 15, 2016}

\begin{document}

\maketitle

\section{Objective}
Categorize crystal structure of the samples previously polished, using optical microscopy.

\section{Samples}
\begin{itemize}
 \item 1018 Steel(Annealed)
 \item 1018 Steel(As Received)
 \item Ti 6-4(Annealed)
\end{itemize}

\section{Procedure}

\subsection{Polishing}
After getting our sample back from last lab, we discovered an oxidation layer that had formed on the polished sample. In order to fix it, we polished our sample once again - documentation is recorded in the previous lab entry. This second polishing reduced the marring of the sample to an acceptable level.

\subsection{Etching}
In order to help us view the grains of our sample, Professor Wilson assisted us by etching our polished sample. She used Nitol to etch, with a duration of 30 seconds. This resulted in distinguishable grains. The group viewed the sample under the microscope, setting the Grain Expert sensitivity to 6, which seemed like a reasonable level visually. Results for our etched transverse and longitudinal samples can all be seen in the LabData folder.

\subsection{Hypereutectoid}
Our group also viewed a hypereutectoid with the optical microscope, calculating the percent surface area.

\end{document}