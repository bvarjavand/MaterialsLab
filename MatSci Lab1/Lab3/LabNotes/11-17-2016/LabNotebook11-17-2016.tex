\documentclass{article}
\usepackage[utf8]{inputenc}
\usepackage{graphicx}

\author{Bijan Varjavand}
\title{LabNotebook}
\date{November 17, 2016}

\begin{document}

\maketitle

\section{Introduction}

This day our group measured the Vickers hardnesses of our previously mounted samples, as well as some initial Nylon data collection.
\subsection{Goals}

The goal was to learn more about hardness across different materials and develop a better intuition about materials science. We also wanted to learn about polymer synthesis.
\section{Purpose}

Vickers hardness, opposed to Rockwell hardness, has a different scale and method of measurement. It utilizes microscopy and a constant distance between the indenter and stage at certain magnifications to normalize values. Aluminum data may show different trends, or not - regardless we will learn about correlation between materials and hardness.
\section{Setup}

The sample pucks were recovered from storage the previous week. The TAs prepared the necessary reactants for the nylon 6,10 synthesis.
\subsection{Materials}

1018 Steel annealed, 6061 Al annealed, Nylon 6,10.
\subsection{Tools}

An microscope with a stage to hold samples, with an indenter also mounted. Pipettes, mass scales, stir rod.
\section{Procedure}

The puck was placed into the hardness tester. After indentation, height and width data of the indent was taken down along with hardness data. We ran the hardness tester at 200gf for Aluminum.\\

After mixing 0.6g of HDMA in 20mL of water, we prepared 0.6g of sebacocyl chloride dissolved in 20mL of water.

The first solution was poured over the second one, and a film was allowed to form (the nylon 6-10). A rod was used to pull stands from the film.
\section{Results}

\begin{table}[h]
\centering
\begin{tabular}{|| c | c | c | c | c | c ||}
\hline
Material & D1 & D2 & HV\\
\hline
\hline
Al round 1 & 103.82 & 99.96 & 36\\
\hline
Al round 2 & 88.14 & 85.64 & 49\\
\hline
Al round 3 & 77.95 & 79.32 & 64\\
\hline
Al round 4 & 71.75 & 66.87 & 77\\
\hline
\end{tabular}
\end{table}

\section{Observations}

The puck that we mounted our Aluminum samples in gave similar results. The reason that we only have HV values is because the Aluminum was too soft to be measured on other hardness scales accurately. Still, the same relationship with increasing hardness due to cold rolling can be seen.\\

The nylon synthesis was successful, and resultant polymer was saturated with water after it was made. After waiting a week, we observed the polymer after it had dried up into a sponge-like material, and it was not very strong.
\end{document}