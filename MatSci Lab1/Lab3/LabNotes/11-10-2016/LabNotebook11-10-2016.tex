\documentclass{article}
\usepackage[utf8]{inputenc}
\usepackage{graphicx}

\author{Bijan Varjavand}
\title{LabNotebook}
\date{November 10, 2016}

\begin{document}

\maketitle

\section{Introduction}

This day our group measured the Vickers hardnesses of our previously mounted samples.
\subsection{Goals}

The goal was to learn how to use the Vickers hardness tester as well as to learn more about different hardness scales and develop a better intuition about materials science in general.
\section{Purpose}

Vickers hardness, opposed to Rockwell hardness, has a different scale and method of measurement. It utilizes microscopy and a constant distance between the indenter and stage at certain magnifications to normalize values. Experience with this process will help students learn more about materials science.
\section{Setup}

The sample pucks were recovered from storage the previous week.
\subsection{Materials}

1018 Steel annealed, 6061 Al annealed.
\subsection{Tools}

An microscope with a stage to hold samples, with an indenter also mounted.
\section{Procedure}

The puck was placed into the hardness tester. After appropriate software had been launched, we focused on a specific portion of the sample, then set the force of the impact. After indentation, height and width data of the indent was taken down along with hardness data. We ran the hardness tester at 500gf for Steel using only 50gf for the thinnest sample.

We did not have enough time to measure all of our data - only Steel is recorded below.
\section{Results}

\begin{table}[h]
\centering
\begin{tabular}{|| c | c | c | c | c | c ||}
\hline
Material & D1 & D2 & HV & HRB & 15T\\
\hline
\hline
Steel round 1 & 97.65 & 96.97 & 192 & 92.6 & 90.8\\
\hline
Steel round 2 & 86.06 & 83.53 & 128 & 71.8 & 84.3\\
\hline
Steel round 3 & 62.01 & 59.36 & 247 & 21.8 & N/A\\
\hline
Steel round 4 & 61.53 & 56.78 & 265 & 24.7 & N/A\\
\hline
\end{tabular}
\end{table}

\section{Observations}

We can see how hardness values change as the sample is rolled. This can be seen by the decrease in indent left - as both dimensions decrease as number of rolls increases. Some hardness values were out of range of specific scales and were thus listed as N/A on those scales.
\end{document}