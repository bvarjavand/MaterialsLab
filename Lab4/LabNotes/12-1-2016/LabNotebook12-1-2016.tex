\documentclass{article}
\usepackage[utf8]{inputenc}
\usepackage{graphicx}

\author{Bijan Varjavand}
\title{LabNotebook}
\date{December 1, 2016}

\begin{document}

\maketitle

\section{Introduction}

This day our group synthesized nylon 6. It is a stronger form of nylon than the 6,10 made at the end of lab 3. We also made "slime" which was just elmer glue, water, and Borax crosslinker.
\subsection{Goals}

The goal was to learn more about different polymers and crosslinking.
\section{Purpose}

Polymer science is a huge and booming field in materials science, rapidly taking over many industries. Knowledge of polymers is crucial for materials scientists researching relevant topics, and this lab introduces basic concepts needed for a broad understanding.
\section{Setup}

The TAs prepared the reactants for both the nylon 6 reaction as well as the slime reaction. The preparation for nylon 6 was much more in-depth. The TA for our group, Mike, heated a mixture of 5g Caprolactam, 0.1g POE (polyoxyethylene), and 3 drops of N-acetylcaprolactam. One the mass stopped bubbling and started to simmer, NaH was added.
\subsection{Materials}

Borax, glue, nylon 6
\subsection{Tools}

Only basic lab tools such as a hot plate and stir rod were needed for this lab.
\section{Procedure}

After NaH was added to the nylon 6 precursor solution, fibers were pulled from the molten polymer. These fibers cooled as they were drawn, and could be pulled out for very long distances (~5-6 meters). Once drawn and cut into sections of similar thicknesses, we measured each portion and tested their tensile properties on the tension tester.\\

Meanwhile, creating the slime was much simpler. The glue and water were mixed in a 1:1 mixture, 1/4 of a cup and 15 mLs water. This stock was poured into 3 different reaction cups, and a different coloring was added to each cup for differentiation. 10, 15, and 20 mL of Borax was added to each cup respectively, changing the amount of crosslinking activity in each cup.
\section{Results}

Quantitative measurements of the slime showed that increasing the amount of crosslinking made the material stronger and less drippy, implying that the rheology changed by increasing the force needed for plastic flow. This can be explained since less crosslinker would hold together the polymer more weakly.

The nylon 6 thickness measurements are below. Tensile test data was collected by the TAs and is shown in the lab report.
\begin{table}[h]
\centering
\begin{tabular}{|| c | c ||}
\hline
Small section($\mu m$) & Large Section($\mu m$)\\
\hline
\hline
91.94 & 202.59\\
\hline
83.14 & 200.22\\
\hline
94.17 & 203.47\\
\hline
97.52 & 202.74\\
\hline
\end{tabular}
\end{table}

\section{Observations}

The nylon synthesis was successful, and resultant polymer had lengths of somewhat uniform thickness. Data collected on the tension tester shows vast differences with the literature value, as would be expected. The entire synthesis process, especially the NaH concentration, greatly affects the property of the nylon 6.\\

The slime was a very good visual example of the effects of crosslinking. One could tell at a glance which slimes had which concentrations of Borax.
\end{document}