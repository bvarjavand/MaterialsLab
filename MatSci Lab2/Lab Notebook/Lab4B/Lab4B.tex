\documentclass{article}
\usepackage[utf8]{inputenc}
\usepackage{graphicx}

\author{Bijan Varjavand}
\title{LabNotebook}
\date{April 18, 2017}

\begin{document}

\maketitle

\section{Objectives}

The objective for our group this day was to begin creating two different transistors - two PQT-12 transistors and two ZTO transistors.

\section{Setup}

Different students were available for this lab which worked in Professor Katz's lab. They were available to guide us. Carbon paint was also prepared to use to pain the electrodes.

\subsection{Materials}

The two materials we used for our transistors were PQT-12 and ZTO. The class of transistor these materials make is that of a MOSFET. We had silicon squares prepared as a semiconductor.

\subsection{Tools}

There was a spin coating machine available, as well as a hot plate.

\section{Procedure}

We practiced drawing carbon paint electrodes on a glass slide before painting two of our silicon squares. We then drew a box around the middle before drip coating the surface with PQT-12. It was placed on a hot plate to dry.\\
For the ZTO transistor, we spin coated each silicon square with ZTO and then cleaned up the edges with 2-methoxyethanol. We then put it in a furnace.

\section{Results}

This lab resulted in 4 prepared transistors.

\section{Observations}

The process of making the transistors was actually simpler than I predicted. This is because the procedure was so well documented.

\end{document}