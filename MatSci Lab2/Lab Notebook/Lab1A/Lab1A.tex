\documentclass{article}
\usepackage[utf8]{inputenc}
\usepackage{graphicx}

\author{Bijan Varjavand}
\title{LabNotebook}
\date{January 31, 2017}

\begin{document}

\maketitle

\section{Objectives}

The goal this lab was to recognize and verify specific equations, specifically Ohm's law.

\section{Setup}

Samples were provided by the lab, which were an assortment of many different types of rod.
\subsection{Materials}

Brass, Titanium, and Aluminum were the three materials our group chose.\\
Their diameters were 3.02mm, 3.05, and 3.05mm respectively
\subsection{Tools}

Our group used power sources and multimeters for our measurements.
\section{Procedure}

We set up a 4 point probe measurement, combining an ammeter and voltmeter.\\
Initially, we varied voltage and recorded the resulting current.\\
Then, we held current constant, and measured the change in voltage due to separation distance of the measurement probes.
\section{Results}
Below.
\begin{figure}[h!]
	For our first sample, brass, we kept length constant at 21 cm and changed the voltage:
	
	\centering
	\begin{tabular}{|c|c|c|}
	\hline	
	\ & Voltage & Current\\
	\hline
	\hline
	1 & 0.058 & 0.031\\
	\hline
	2 & 0.194 & 0.117\\
	\hline
	3 & 0.281 & 0.164\\
	\hline
	4 & 0.395 & 0.232\\
	\hline
	5 & 0.521 & 0.307\\
	\hline
	6 & 0.738 & 0.436\\
	\hline
	7 & 1.444 & 0.856\\
	\hline
	8 & 2.681 & 1.592\\
	\hline
	9 & 3.359 & 2.000\\
	\hline
	\end{tabular}
\end{figure}
\begin{figure}[h!]
	We then measured the voltage change due to varying length across 3 different materials, shown below (with constant current of 1 Amp:
	\centering
	\begin{tabular}{|c|c|c|c|c|c|c|}
		\hline
		\ & All & Brass & \ & Titanium & \ & Aluminum\\
		\hline
		\ & Length(cm) & Voltage(mV) & \ & Voltage(mV)	& \ &  Voltage(mV)\\
		\hline
		\hline
		1 & 2 & 0.161 & \ & 1.412 & \ & 0.206\\
		\hline
		2 & 4 & 0.323 & \ & 2.475 & \ & 0.330\\
		\hline
		3 & 6 & 0.468 & \ & 3.806 & \ & 0.441\\
		\hline
		4 & 8 & 0.630 & \ & 4.946 & \ & 0.565\\
		\hline
		5 & 10 & 0.789 & \ & 6.381 & \ & 0.675\\
		\hline
		6 & 12 & 0.937 & \ & 7.556 & \ & 0.781\\
		\hline
		7 & 14 & 1.108 & \ & 8.846 & \ & 0.806\\
		\hline
		8 & 16 & 1.245 & \ & 10.048 & \ & 0.972\\
		\hline
		9 & 18 & 1.394 & \ & 11.350 & \ & 1.063\\
		\hline
	\end{tabular}
\end{figure}

\section{Observations}

The equations we have gone over in class to characterize this phenomenon predict the observed behavior accurately.
\end{document}