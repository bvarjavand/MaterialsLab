\documentclass{article}
\title{Lab 1A: Conductivity}
\author{Bijan Varjavand}
\date{1/31/2017}
\begin{document}
\maketitle
\section{Data, part 1}
We measured 3 rods: Brass, Titanium, and Aluminum
Their diameters were 3.02mm, 3.05, and 3.05mm respectively
\begin{figure}
	For our first sample, brass, we kept length constant at 21 cm and changed the voltage:
	
	\centering
	\begin{tabular}{|c|c|c|}
	\hline	
	\ & Voltage & Current\\
	\hline
	\hline
	1 & 0.058 & 0.031\\
	\hline
	2 & 0.194 & 0.117\\
	\hline
	3 & 0.281 & 0.164\\
	\hline
	4 & 0.395 & 0.232\\
	\hline
	5 & 0.521 & 0.307\\
	\hline
	6 & 0.738 & 0.436\\
	\hline
	7 & 1.444 & 0.856\\
	\hline
	8 & 2.681 & 1.592\\
	\hline
	9 & 3.359 & 2.000\\
	\hline
	\end{tabular}
\end{figure}
\begin{figure}
	We then measured the voltage change due to varying length across 3 different materials, shown below (with constant current of 1 Amp:
	\centering
	\begin{tabular}{|c|c|c|c|c|c|c|}
		\hline
		\ & All & Brass & \ & Titanium & \ & Aluminum\\
		\hline
		\ & Length(cm) & Voltage(mV) & \ & Voltage(mV)	& \ &  Voltage(mV)\\
		\hline
		\hline
		1 & 2 & 0.161 & \ & 1.412 & \ & 0.206\\
		\hline
		2 & 4 & 0.323 & \ & 2.475 & \ & 0.330\\
		\hline
		3 & 6 & 0.468 & \ & 3.806 & \ & 0.441\\
		\hline
		4 & 8 & 0.630 & \ & 4.946 & \ & 0.565\\
		\hline
		5 & 10 & 0.789 & \ & 6.381 & \ & 0.675\\
		\hline
		6 & 12 & 0.937 & \ & 7.556 & \ & 0.781\\
		\hline
		7 & 14 & 1.108 & \ & 8.846 & \ & 0.806\\
		\hline
		8 & 16 & 1.245 & \ & 10.048 & \ & 0.972\\
		\hline
		9 & 18 & 1.394 & \ & 11.350 & \ & 1.063\\
		\hline
	\end{tabular}
\end{figure}
\end{document}