\documentclass{article}
\usepackage[utf8]{inputenc}
\usepackage{graphicx}

\author{Bijan Varjavand}
\title{LabNotebook}
\date{November 3, 2016}

\begin{document}

\maketitle

\section{Introduction}

This day our group tested the hardness of all of our samples which we cut from the cold roller.
\subsection{Goals}

The goal was both to learn about using the Rockwell hardness tester as well as the properties that it measured.
\section{Purpose}

A crucial property of materials is their hardness, and learning about machinery and the value itself enhances the qualifications of each student as a materials scientist.
\section{Setup}

The samples for testing were recovered from storage the week before.
\subsection{Materials}

1018 Steel annealed, 6061 Al annealed.
\subsection{Tools}

A machine in the lab that heats and compresses polymers around a sample to form a puck. Rockwell hardness tester.
\section{Procedure}

We loaded each sample in a puck side by side, except for our thinnest sample which was loaded with another group's puck due to lack of space on our own. After the pucks were made, their hardness was measured with the Rockwell machine (after calibration).
\section{Results}

Calibration data of a control piece of Al:
\begin{table}[h]
\centering
\begin{tabular}{|| c | c | c | c | c | c ||}
\hline
Sample Hardness(HRB) & 57.0 & 57.8 & 57.6 & 57.0 & 58.3\\
\hline
\end{tabular}
\end{table}

Hardness values can be found in the Lab Report.
\section{Observations}

Hardness of the materials changed with cold rolling - more cold rolled samples exhibited higher hardness across both Al and Steel.
\end{document}