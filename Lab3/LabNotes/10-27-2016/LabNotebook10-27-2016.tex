\documentclass{article}
\usepackage[utf8]{inputenc}
\usepackage{graphicx}

\author{Bijan Varjavand}
\title{LabNotebook}
\date{October 27, 2016}

\begin{document}

\maketitle

\section{Introduction}

This day our group cold rolled all of our samples. We then took dimensional measurements on our samples.
\subsection{Goals}

The goal was to generate data about the effects of cold rolling on different materials.
\section{Purpose}

Cold rolling is a common materials science processing step, and understanding of it will give students a deeper understanding of the field.
\section{Setup}

The samples were prepared beforehand as rectangular prisms. The entire group took 2 vices and hacksaws to an external shop that had a cold rolling machine.
\subsection{Materials}

1018 Steel annealed, 6061 Al annealed.
\subsection{Tools}

Cold rolling machine, calipers, hacksaw.
\section{Procedure}

We cold rolled our samples to lower and lower thicknesses, cutting a piece of at various thicknesses for analysis. The dimensions of each piece were recorded.
\section{Results}

Each round of rolling was a few rolls, enough to have visual thickness change.
\begin{table}[h]
\centering
\begin{tabular}{|| c | c | c ||}
\hline
\ & Al 6061(mm) & Steel 1018(mm)\\
\hline
\hline
0 rounds & 7.93x25.33 & 7.9x31.69\\
\hline
1 round & 6.14x25.70 & 6.45x32.17\\
\hline
2 rounds & 4.80x26.11 & 4.76x33.58\\
\hline
3 rounds & 2.25x26.20 & 2.25x36.10\\
\hline
4 rounds & 1.10x27.60 & 1.40x37.50\\
\hline
\end{tabular}
\end{table}
\section{Observations}

Rolling affected dimensions in the way we expected, but affected each material differently. This is due to the elasticity difference between the two metals.
\end{document}