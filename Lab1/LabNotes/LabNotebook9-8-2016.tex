\documentclass{article}
\usepackage[utf8]{inputenc}
\usepackage{graphicx}

\author{Bijan Varjavand\\Group 2d}
\title{Lab Notebook}
\date{September 8, 2016}

\begin{document}

\maketitle

\section{Objective}
Categorize crystal structure of 3 samples, using optical microscopy and XRD.

\section{Samples}
\begin{itemize}
 \item 1018 Steel(Annealed)
 \item 1018 Steel(As Received)
 \item Ti 6-4(Annealed)
\end{itemize}

\section{Procedures}

\subsection{Optical Microscope}
We used the LAS v4.7 software to capture optical images of our samples. The first step was to turn on the bright field setting on the lowest magnification. After focusing and moving the sample to a clean part, we increase the magnification and refocused. We actually also needed to increase the brightness. Then we selected the "acquire image" button and saved it.

For processing of the image, we went into the Analyze tab and selected "Grain Expert", then "Identify grain boundary." This tool used a sensitivity setting to classify boundaries, which we decided on by eye to be 6(recognized most grains without overrecognition). After, we used the ASTM standard to generate calculations for grain size, with the intercept method. We specifically viewed transverse and longitudinal as-received 1018 steel. 

We saved multiple images, which can be found in the "LabData" folder. The images saved had the GrainExpert overlay, as well as the ASTM-standardized bars for calculating grain number.

\subsection{Polishing}
To prepare our annealed 1018 steel samples for optical microscopy, we manually sanded 2 samples to remove the layer of oxidation on the surface. We purposely sanded different sides in order to obtain a polished surface on both a transverse and longitudinal side. Once they were roughly polished, we packed epoxy around them and pressurized them at 200$^o$C to generate epoxy pucks. Once the pucks were ready, we polished them with a specific machine with its own specific settings. The machine used smaller and smaller sizes of diamond solution.

\subsection{XRD}
While waiting for our epoxy pucks to cool, we had enough time to generate XRD peaks from Ti6-4. This data is also in the LabData folder. The procedure consisted of setting a starting angle, ending angle, and step size. The wavelength was pre-set.



\end{document}