\documentclass{article}
\usepackage[utf8]{inputenc}
\usepackage{graphicx}

\author{Bijan Varjavand}
\title{LabNotebook}
\date{October 6, 2016}

\begin{document}

\maketitle

\section{Introduction}

This day, our group did the tension test and the SEM imaging.
\subsection{Goals}

Generate data for tension testing and generate images from SEM.
\section{Purpose}

These data will provide opportunity for analysis, which can provide insight into material properties.
\section{Setup}

We used our previously prepared samples for the SEM imaging which were created with the charpy impact tester.
The samples for tension testing were prepared beforehand.
\subsection{Materials}

1018 steel as received, 1018 steel annealed, 6061 Al as received, and 6061 Al annealed, in the dogleg shape.
\subsection{Tools}

We used an extensometer during our tensile testing in order to record strain data. We also used the SEM and the tension tester.
\section{Procedure}

\subsection{Tension Test}

After sanding our samples to remove oxide layers which may interfere with data, we put them into the tension tester. After attaching the extensometer, we ran the software with preset values. After running the program for each sample, the collected data was stored.
\subsection{SEM}

We used carbon tape to attach our samples to the viewing plate, and the TA took images and we stored it.
\section{Results}

The resulting data can be found in the LabData directory for Lab2.
\section{Observations}

Since steel was stronger it made a louder noise.
\end{document}