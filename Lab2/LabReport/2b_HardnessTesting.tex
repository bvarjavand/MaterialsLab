\documentclass{article}
\usepackage[utf8]{inputenc}
\usepackage{times}
\usepackage{graphicx}
\usepackage[margin=0.75in]{geometry}
\usepackage{hyperref}
\renewcommand{\baselinestretch}{1.5}

\author{Bijan Varjavand}
\title{Lab 2b: Hardness Testing\\Group 2d}

\begin{document}

\maketitle

\clearpage

One important aspect of a unit cell is its burgers vector and dense plane. These values indicate the direction of most dense packing. Notation for these dense direction is in miller indices (h, k, l, where hkl are orthogonal directional axes). Not only that, but due to symmetry of unit cells, the burgers vector is a family of h, k, and l values. For example, the family \{1 1 0\} includes all hkl values that, when squares are added, equal 1. This is actually the burgers vector for FCC unit cells, and a one is shown below.

One can see that, in the unit cell space, the "dense direction" is easily found as the highest density of atoms in a specific direction(\textbf{Fig2}). This is also the direction that slips form across most easily due to the close stacking of atoms. This is due to stacking fault energy in that direction being the lowest, as the maximum displacement between atoms is lowest(due to distance between atoms being the lowest as well).

Dislocations are linear imperfections in a crystal structure, and occur along the Burgers vector. The more dislocations are present in a material, the more difficult it is for the material to be bent and shaped. This is due to the dislocations.

The grain sizes for 1018 steel and Aluminum 6061 are isotropic. This implies that their strength is also isotropic, allowing for more practical application. The intensities of the peaks in the XRD spectrum are at values of 2$\theta$ that are accurate. These lab findings give students more confidence in any projects related to mounting, polishing, and visualizing materials. Not only that, but confidence in analyzing XRD data is also elevated.

\clearpage

\begin{figure}
	\centering
	%\includegraphics[scale=0.3]{wow.png}
	\caption{Hardness wow}
\end{figure}
\end{document}