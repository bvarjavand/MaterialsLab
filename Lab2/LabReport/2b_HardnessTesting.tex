\documentclass{article}
\usepackage[utf8]{inputenc}
\usepackage{times}
\usepackage{graphicx}
\usepackage[margin=0.75in]{geometry}
\usepackage{hyperref}
\renewcommand{\baselinestretch}{1.5}

\author{Bijan Varjavand}
\title{Lab 2b: Hardness Testing\\Group 2d}

\begin{document}

\maketitle

\clearpage

Talk about hardness and how it is measured (different scales, units, equations). Describe in more detail the Vickers and Rockwell hardness tests (generally).

Talk about sample prep.

Procedure for Vickers test. Reference tables and figures, including lots of data. Aluminum data, then steel.

Procedure for Rockwell test. Reference tables and figures, including lots of data. Aluminum data, then steel.

Analysis of data, look at average and standard deviation. Compare to literature values. Convert to hardness in YS/TS and compare.

Show results from calibrations? Compare with 2a values.

\clearpage

Appendix:\\

\begin{table}[h]
\centering
\begin{tabular}{||c | c | c | c | c | c||}
	\hline
	\ 	&
	1	&	2	&	3	&	4	&	5\\
	\hline\hline
	Annealed 1018 Steel		&
	120	&	124	&	123	&	123	&	123\\
	Annealed 6061 Aluminum	&
	37	&	39	&	40	&	40	&	40\\
	\hline
\end{tabular}
\caption{Vicars Hardness Data}
\end{table}

\begin{table}[h]
\centering
\begin{tabular}{||c | c | c | c | c | c||}
	\hline
	\ 	&
	1	&	2	&	3	&	4	&	5\\
	\hline\hline
	As-Received 1018 Steel		&
	58.6	&	58.3	&	58.2	&	58.5	&	58.3\\
	Annealed 1018 Steel		&
	32.7	&	36.3	&	35.9	&	35.9	&	35.5\\
	As-Received 6061 Aluminum	&
	36.9	&	38.1	&	40.1	&	39.8	&	38.7\\
	\hline
\end{tabular}
\caption{Rockwell Hardness Data}
\end{table}

\begin{figure}[h]
	\centering
	%\includegraphics[scale=0.3]{wow.png}
	\caption{Images}
\end{figure}
\end{document}