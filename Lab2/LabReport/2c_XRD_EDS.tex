\documentclass{article}
\usepackage[utf8]{inputenc}
\usepackage{times}
\usepackage{graphicx}
\usepackage[margin=0.75in]{geometry}
\usepackage{hyperref}
\renewcommand{\baselinestretch}{1.5}

\author{Bijan Varjavand}
\title{Lab 2b: Hardness Testing\\Group 2d}

\begin{document}

\maketitle

\clearpage

Talk about toughness and how it is measured (different scales, units, equations). Describe in more detail the relationship between toughness and temperature - for FCC and BCC.

Talk about sample prep.

Procedure for charpy toughness test. Reference tables and figures, including lots of data.

Procedure for SEM and EDS. Reference figures.

Analysis of data. Compare to literature values. Ductile to brittle transition(report as temp. range)?

Sources of Error(fluctuations in temperature)?

\clearpage

Appendix:\\

\begin{table}[h]
\centering
\begin{tabular}{||c | c | c | c | c | c||}
	\hline
	\ 							& 1 & 2 & 3 & 4 & 5\\
	\hline\hline
	As-Received 1018 Steel		& 0 & 0 & 0 & 0 & 0\\
	As-Received 6061 Aluminum	& 0 & 0 & 0 & 0 & 0\\
	\hline
\end{tabular}
\caption{Charpy test data}
\end{table}

\begin{figure}[h]
	\centering
	%\includegraphics[scale=0.3]{wow.png}
	\caption{Charpy plots}
\end{figure}

\begin{figure}[h]
	\centering
	%\includegraphics[scale=0.3]{wow.png}
	\caption{SEM images}
\end{figure}
\end{document}